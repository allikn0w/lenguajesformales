\documentclass{article}

\usepackage{lmodern}
\usepackage[T1]{fontenc}
\usepackage[spanish,activeacute]{babel}
\usepackage{mathtools}
\usepackage[utf8]{inputenc}
\usepackage{enumerate}
\usepackage[a4paper, total={7in, 11in}]{geometry}
\usepackage{charter}

\begin{document}
\begin{center}
\large \textbf{Práctica 6}: Lenguajes formales y gramáticas
\end{center}
\textbf{Definición 1.} Una gramática se dice:
\begin{enumerate}[(a)]
\item
    \textit{regular} si cada producción es de la forma:
    $A \rightarrrow a$ o $A \rightarrow aB$ o $A \rightarrow \lambda$ 
    donde $A, B \in N$ y $a \in T$,
\item
    \textit{libre} (o independiente) de contexto si cada producción es de la forma 
    $A \rightarrow \delta$ donde $A \in N$ y $\delta \in (N \cup T)*$
\item
    \textit{sensible} al conexto si cada producción es de la forma
    $aA\beta \rightarrow \alpha\delta\beta$ donde $A \in N, \alpha,\beta \in (N \cup T)*$ y $\delta \in (N \cup T)+$,
\item
    \textit{estructurada} por frases o irrestricta si no tiene restricciones sobre la forma de sus producciones, es decir
   si son de la forma
    \[ \alpha \rightarrow \delta \text{\quad donde \quad} \alpha \in (N \cup T)* - T* y \delta \in (N \cup T)* \]
\end{enumerate}

\end{document}

